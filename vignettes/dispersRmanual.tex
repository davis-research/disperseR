\documentclass{article}
\usepackage{natbib}
\title{DisperseR: Calculating Seed Dispersal In R}
\author{Samantha L. Davis}
\usepackage{Sweave}
\begin{document}
\maketitle
\Sconcordance{concordance:dispersRmanual.tex:dispersRmanual.Rnw:%
1 4 1 1 0 31 1 1 2 1 0 2 1 11 0 1 1 12 0 1 2 5 1}


\section{Introduction}

This is a small package intended to help users calculating seed dispersal in R. Although the R base machinery is capable of doing so, this package streamlines the process and enables you to focus more on the important aspects of data analysis instead of data generation or clean-up.

This code operates as follows. Ideally, you'll need a dataframe that contains the following data: (x,y) coordinates of each tree and seedling in a plot; and dbh measurements of any tree large enough. A tree is any individual that can be measured for diameter at breast height, and all trees are assumed to be reproductively active; a seedling is any individual that is new in the calendar year.

Spatial seed dispersal is characterized by a single equation,

\begin{equation}
\label{eq:dispersal}
R_i = STR * \sum\limits_{k=1}^T\left( \frac{DBH_k}{30}\right) ^2 e^{-Dm_{ik}^3} * \left( \frac{1}{n}\right)
\end{equation}

where \textit{n} is a normalizer function that standardizes the equation to values between 0 and 1,

\begin{equation}
n = \int\limits_{0}^\infty e^{-Dm_{ik}^3} \nonumber
\end{equation}

and where \textit{STR} is the standardized number of tree recruits, \textit{DBH} is the diameter at breast height, \textit{D} is a species-specific parameter estimated by this equation, and \textit{m} is the distance between the measured point \textit{i} and adult tree \textit{k}, summed over each adult tree (\textit{k}=1 to \textit{T} adult trees). These equations were originally established by \citet{Ribbens1994}, in an experiment where seedling per $m^2$ along a belt transect were correlated to the number and size of any adults within a $20 m$ radius.

The first piece of the equation, containing STR, establishes the number of recruits produced for a tree of a standard DBH (30cm), and the second piece of the equation establishes the mean density of recruits found in a $1 m^2$ quadrat centered at \textit{m} distance away from the parent tree. Finally, $\frac{1}{n}$ serves as a normalizer to standardize the equation across species.

The parameters \textit{STR} and {D} are both needed by SORTIE-ND, an individual tree neighborhood dynamics forest gap model (say that five times fast!), to calculate seed dispersal for target species in its simulations. SORTIE-ND, unfortunately, does not come packaged with a magic bullet that offers species-specific parameters, and therefore, we must parameterize the model ourselves. This package is intended to help create estimates of both \textit{STR} and \textit{D} quickly, so that other parameters may be addressed.

What follows is a list of functions alongside example usage. To start, you must import or generate a plot map of all trees in a given area. This plot map must include a species identifier, an x coordinate, a y coordinate, and DBH (or NA) for each individual.

We can generate a sample plot easily with generatePlotMap(). As you can see below, this function generates a plot map with NA's for seedlings and actual values of DBH for adult trees. See ?generatePlotMap() for information on how to customize your random plot map.
\begin{Schunk}
\begin{Sinput}
> library(disperseR)
> myplot <- generatePlotMap()
> head(myplot)
\end{Sinput}
\begin{Soutput}
  species          x        y dbh
1       1 54.6869423 54.22306  NA
2       1  0.6989224 48.65458  NA
3       1 36.4854372 70.25562  NA
4       1 11.3466584 19.18422  NA
5       1 65.2956505 14.86108  NA
6       1 74.8151938 27.04914  NA
\end{Soutput}
\begin{Sinput}
> tail(myplot)
\end{Sinput}
\begin{Soutput}
   species         x         y       dbh
55       3 21.576138 41.318587 17.422166
56       3 12.440960 38.173444  8.465424
57       3 11.524822 70.025328 91.295164
58       3 40.583205 17.483600 88.223219
59       3  7.775619  5.392415  2.301222
60       3 55.271426 98.093571 65.017483
\end{Soutput}
\end{Schunk}



\bibliographystyle{sty/ecology}
\bibliography{disperseRmanual}
\end{document}
